\RequirePackage{subfigure}
\RequirePackage{graphicx}
\RequirePackage{color,graphics}
\RequirePackage{caption}
\RequirePackage{amsmath} % This is for numbering equations
\RequirePackage{listings}
\RequirePackage[ansinew]{inputenc} % This is to be able to type � instead of \'a
\RequirePackage[spanish]{babel} % This is to get Spanish hyphenation



% The scale for figures

\newcommand{\scale}{0.4}
\newcommand{\linescale}{0.6}
\newcommand{\smallscale}{0.5}

% ------------------- Other definitions

\newcommand{\ejs}{\emph{EJS}}
\newcommand{\Ejs}{\emph{Easy Java Simulations}}
\newcommand{\osp}{\emph{OSP}}
\newcommand{\Osp}{\emph{Open Source Physics}}

\newcommand{\file}[1]{\textbf{#1}} % the name of a file or directory
\newcommand{\code}[1]{{\ttfamily #1}} % a variable

\newcommand{\url}[1]{{\ttfamily #1}} % a variable
\newcommand{\lit}[1]{{\sl #1}}  % literal text
%\newcommand{\lit}[1]{``#1''}  % literal text
\newcommand{\link}[1]{\emph{$<$#1$>$}}  %URL Links
\newcommand{\note}[1]{\begin{quote}\small #1\end{quote}}  % a big note with color background

%Use special environments for listings, exercises, problems, and projects.
\newenvironment{listing} {\ttfamily\small}{}
%exerciseEnv is used to define the exercise environment
\newtheorem{exerciseEnv}{\hspace{-\parindent} Ejercicio}[chapter]
%The exercise environment adds a box at the end of the exercise narrative to indicate the end of the exercise.
\newenvironment{exercise}[1][]{\begin{exerciseEnv}{\textbf{#1}\newline}}{\boldmath $\Box$ \end{exerciseEnv}}
%problems and projects appear at the end of each chapter
\newtheorem{problem}{\hspace{-\parindent} Problema}[chapter]
\newtheorem{project}{\hspace{-\parindent} Proyecto}[chapter]

% environments for long listings
\definecolor{background}{rgb}{1,1,1}
\definecolor{sepcolor}{rgb}{0.90,0.90,0.90}
\lstset{language=Java,
linewidth=30pc,
showstringspaces=false,
commentstyle=\upshape,
stringstyle=\ttfamily,
%basicstyle=\small,
basicstyle=\scriptsize\ttfamily\baselineskip10pt, breaklines=true, keywordstyle=\upshape,
breakatwhitespace=true, morecomment=[is]{/*}{*/}, moredelim=**[is][\hfill\bfseries]{/*+}{+*/},
moredelim=[is][\emph]{+-}{-+}, moredelim=[is][\emph]{/*-}{-*/}} 